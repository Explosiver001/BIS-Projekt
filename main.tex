\documentclass[a4paper, 11pt]{article}
\usepackage[a4paper, top=3cm,left=2cm,text={17cm,24cm}]{geometry}
\usepackage[utf8]{inputenc}
\usepackage{times}
\usepackage[czech]{babel}
\usepackage{listings}
\usepackage{url}
\begin{document}
    \begin{titlepage}
        \begin{center}
            \Huge{\scshape{Vysoké učení technické v Brně}\\}
            \huge{\scshape Fakulta informačních technologií\\}
            \vspace{\stretch{0.382}}

            \LARGE{Bezpečnost informačních systémů\\}
            \Huge{Projekt \-- Starověké zlo se probouzí}

            \vspace{\stretch{0.618}}
        \end{center}
        
        \Large{\today \hfill Michal Novák (xnovak3g)}
    \end{titlepage}


\section{Zmapování Středozemě}
\subsection{Nalezené servery}

\begin{table}[!ht]
    \centering
    \begin{tabular}{|l|c|l|}
         \hline
         Název & IP & Služby \\ 
         \hline\hline
         bis2024\_rivendell\_1 & 10.89.1.3 & SSH - port 22\\ 
         \hline
         bis2024\_isengard\_1 & 10.89.1.4 & HTTP - port 80\\
         \hline
         bis2024\_edoras\_1 & 10.89.1.5 & FTP - port 21\\
         \hline
         bis2024\_mirkwood\_1 & 10.89.1.6 & HTTP -port 80\\
         \hline
         bis2024\_admin\_1 & 10.89.1.2 & - \\
         \hline
         
    \end{tabular}
    \caption{Caption}
    \label{tab:my_label}
\end{table}

\subsection{Nalezené zranitelnosti}
\subsubsection{bis2024\_rivendell\_1}

SHELLSHOCK

\subsubsection{bis2024\_isengard\_1}
\subsubsection{bis2024\_edoras\_1}
\subsubsection{bis2024\_mirkwood\_1} 

SQL injection


\subsubsection{bis2024\_admin\_1}



\section{Nalezení tajemství}
\subsection{Tajemství D}
Pro objevení tajemství D jsem vycházel z nápovědy D. Tajemství se nachází na serveru \uv{bis2024 mirkwood 1} is IP \uv{10.89.1.6}. Na tomto serveru běží HTTP server na portu 80. 

Dotazem pomocí příkazu \texttt{curl} na \url{http://10.89.1.6/} načteme domovskou stránku HTTP serveru, na které jsou zřetelné 2 odkazy \url{/authentication.html} a \url{/upload.html}, z nichž je pro tajemství D klíčový odkaz \url{/authentication.html}. Dotaz na \url{http://10.89.1.6/authentication.html} odhalí, že se jedná o autentizační formulář, který je odeslán na \url{authenticate.php}. Po odeslání formuláře pomocí \texttt{curl} s daty \verb|-F "username=name" -F "password=pass' hello"| na \url{authenticate.php} se zobrazí chybová hláška SQLite3 databáze. Po upravení hodnoty s heslem ve formuláři na \linebreak \verb|-F "password=pass' OR '1'='1"| je patrné, že php script neošetřuje útoky typu SQL Injection a~zároveň, že po přihlášení došlo k přesměrování na \url{/list.php?id=0}. Id v URL je v tomto případě použito pro získání dat z databáze, čehož je možno zneužít použitím \verb|UNION|. Nejdříve je ale potřeba zjistit, kolik sloupců je ve výsledku dotazu z databáze, což lze pomocí \verb|ORDER BY| postupným zkoušením. Poté už jen stačí položit dotaz na data z předpokládané tabulky \verb|users| dosazením za \url{/list.php?id=} URL-encoded příkaz \linebreak \verb|2 UNION SELECT username,password,NULL from users--|, čímž se odhalí tajemství D.




\end{document}